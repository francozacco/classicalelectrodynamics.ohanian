\documentclass[11pt]{article}
\usepackage{amssymb}
\usepackage{amsthm}
\usepackage{enumitem}
\usepackage{amsmath, physics}
\usepackage{bm}
\usepackage{adjustbox}
\usepackage{mathrsfs}
\usepackage{graphicx}
% \usepackage{siunitx}
\usepackage{physunits}
\usepackage[mathscr]{euscript}

\title{\textbf{Solved selected problems of Classical Electrodynamics - Hans Ohanian}}
\author{Franco Zacco}
\date{}

\addtolength{\topmargin}{-3cm}
\addtolength{\textheight}{3cm}

\newcommand{\Nat}{\mathbb{N}}
\newcommand{\Z}{\mathbb{Z}}
\newcommand{\Q}{\mathbb{Q}}
\newcommand{\R}{\mathbb{R}}
\newcommand{\diam}{\text{diam}}
\newcommand{\cl}{\text{cl}}
\newcommand{\bdry}{\text{bdry}}
\newcommand{\inter}{\text{int}}
\newcommand{\hatx}{\bm{\hat{x}}}
\newcommand{\haty}{\bm{\hat{y}}}
\newcommand{\hatz}{\bm{\hat{z}}}
\newcommand{\hatr}{\bm{\hat{r}}}
\newcommand{\hatn}{\bm{\hat{n}}}
\newcommand{\hatrho}{\bm{\hat{\rho}}}
\newcommand{\hatphi}{\bm{\hat{\phi}}}
\newcommand{\hattheta}{\bm{\hat{\theta}}}
\newcommand{\vecx}{\bm{x}}
\newcommand{\varep}{\varepsilon}


\theoremstyle{definition}
\newtheorem*{solution*}{Solution}
\renewcommand*{\proofname}{\bf{Solution}}

\begin{document}
\maketitle
\thispagestyle{empty}

\section*{Chapter 4 - Dielectrics}

\subsection*{Exercises}

\begin{proof}{\textbf{Exercise 1.}}
From the Gauss' law for Dielectrics in its integral form we know that
\begin{align*}
    \int \div \bm{D}~dV = \int \bm{D}\cdot d\bm{S} = 4\pi \int \rho_F~dV
\end{align*}
But in this case, since we are assuming a pillbox where the sides of the box
can be ignored as $h \to 0$, Gauss' equation becomes
\begin{align*}
    \int \bm{D}\cdot d\bm{S} = AD_{2\perp} - AD_{1\perp} = 4\pi \sigma_F A
\end{align*}
Where $A$ is the area of the top and bottom sides of the pillbox, and we
assumed that $\sigma_F$ is constant inside the pillbox. Therefore
the boundary condition becomes
\begin{align*}
    D_{2\perp} - D_{1\perp} = 4\pi \sigma_F
\end{align*}
\end{proof}

\cleardoublepage
\begin{proof}{\textbf{Exercise 2.}}
We want to compute the integral $\int \bm{E}\cdot d\bm{S}$ so taking a pill box
that is half in the disk-cavity and half in the dielectric and assuming the
width of the pill box $h$ tends to 0 we get that
\begin{align*}
    \int \bm{E}\cdot d\bm{S} = A(E_{\perp \text{ext}} - E_{\perp \text{int}})
    = A(E_0 - \varepsilon E_0)
    =AE_0(1 - \varepsilon)
\end{align*}
Where $A$ is the area of the face of the pill-box.
But also by Gauss' law we have that
\begin{align*}
    \int \bm{E}\cdot d\bm{S} = 4\pi\sigma_P A
\end{align*}
where we used $\sigma_P$ since there are no free charges.
\\
Therefore the (bound) surface charge density is given by
\begin{align*}
    \sigma_P = \frac{E_0}{4\pi}(1 - \varepsilon) 
\end{align*}
Let us be sufficiently near the surface of the cavity, there the edge effects
are unimportant and we may regard the surface as infinite. So the electric field
generated by such an infinite plane has a magnitude $2\pi \sigma_P$ i.e.
\begin{align*}
    2\pi\sigma_P = \frac{E_0}{2}(1 - \varepsilon)
\end{align*}
But we must multiply this expression by two since we have two surfaces (each
face of the disk) and hence the surface charge density contributes an
electric field of $E_0(1 - \varepsilon)$ to the electric field inside the
cavity.
\end{proof}

\cleardoublepage
\begin{proof}{\textbf{Exercise 3.}}
Equation (42) states that
\begin{align*}
    -E_0R\cos\theta + \sum_{l=1}^\infty A_l \frac{P_l(\cos\theta)}{R^{l+1}}
    = \sum_{l=0}^\infty B_lR^lP_l(\cos\theta)
\end{align*}
If equation (42) is to be valid for all values of $\theta$ then if
$\theta = \pi/2$ we get that
\begin{align*}
    \sum_{l=1}^\infty A_l \frac{P_l(0)}{R^{l+1}}
    &= \sum_{l=0}^\infty B_lR^lP_l(0)
    = B_0 + \sum_{l=1}^\infty B_lR^lP_l(0)
\end{align*}
If $l$ is odd then $P_l(0) = 0$ so we get that
$$B_0 = 0$$
but if $l$ is even $P_l(0) \neq 0$ so we see that
\begin{align*}
    \sum_{\substack{l=1\\ l\text{ even}}}^\infty
    A_l \frac{P_l(0)}{R^{l+1}} - B_lR^lP_l(0) = 
    \sum_{\substack{l=1\\ l\text{ even}}}^\infty
    P_l(0)\bigg(\frac{A_l}{R^{l+1}} - B_lR^l\bigg) &= 0
\end{align*}
So for this to be equal to $0$ must be that $A_l/R^{l+1} - B_lR^{l} = 0$ 
then we see that
\begin{align*}
    \frac{A_2}{R^{3}} &= B_2R^2\\
    \frac{A_4}{R^{5}} &= B_4R^4\\
    &\vdots
\end{align*}
Now, let us multiply equation (42) by $P_n(\mu)$ where $\mu =\cos\theta$ and
let us integrate with respect to $\mu$ between -1 and 1
\begin{align*}
    -E_0R\int_{-1}^1\mu P_n(\mu)~d\mu
    + \sum_{l=1}^\infty \int_{-1}^1 A_l \frac{P_l(\mu)P_n(\mu)}{R^{l+1}}~d\mu
    = \sum_{l=0}^\infty \int_{-1}^1 B_lR^lP_l(\mu)P_n(\mu)~d\mu
\end{align*}
Let $n = 1$ then since $\int_{-1}^1 P_l(\mu)P_n(\mu)~d\mu = 0$ when $l \neq n$
we get that
\begin{align*}
    -E_0R\int_{-1}^1\mu P_1(\mu)~d\mu + \frac{2}{3}\frac{A_1}{R^{2}}
    &= \frac{2}{3}B_1R\\
    -E_0R\int_{-1}^1\mu^2~d\mu + \frac{2}{3}\frac{A_1}{R^{2}}
    &= \frac{2}{3}B_1R\\
    -\frac{2}{3}E_0R + \frac{2}{3}\frac{A_1}{R^{2}}
    &= \frac{2}{3}B_1R\\
    -E_0R + \frac{A_1}{R^{2}} &= B_1R
\end{align*}
Where we used that $\int_{-1}^1 P_l(\mu)P_l(\mu)~d\mu = 2/(2l + 1)$ and that
$P_1(\mu) = \mu$.
\\
If we let $n = 3$ we get that
\begin{align*}
    -E_0R\int_{-1}^1\mu P_3(\mu)~d\mu + \frac{2}{7}\frac{A_3}{R^{4}}
    &= \frac{2}{7}B_3R^3\\
    -E_0R\int_{-1}^1 \frac{5\mu^4 - 3\mu^2}{2}~d\mu
    + \frac{2}{7}\frac{A_3}{R^{4}}
    &= \frac{2}{7}B_3R^3\\
    -E_0R\bigg[\frac{5}{2}\frac{\mu^5}{5}\bigg|_{-1}^1
    - \frac{3}{2}\frac{\mu^3}{3}\bigg|_{-1}^1\bigg]
    + \frac{2}{7}\frac{A_3}{R^{4}}
    &= \frac{2}{7}B_3R^3\\
    -E_0R\bigg[\bigg(\frac{1}{2} - \frac{1}{2}\bigg)
    - \bigg(\frac{1}{2} - \frac{1}{2}\bigg)\bigg]
    + \frac{2}{7}\frac{A_3}{R^{4}}
    &= \frac{2}{7}B_3R^3\\
    \frac{2}{7}\frac{A_3}{R^{4}}
    &= \frac{2}{7}B_3R^3\\
    \frac{A_3}{R^{4}}
    &= B_3R^3
\end{align*}
This process can be continued for each odd value of $l$.
\end{proof}

\cleardoublepage
\begin{proof}{\textbf{Exercise 4.}}
Equation (44) states that
\begin{align*}
    \varepsilon_1\bigg[-E_0\cos\theta
    - \sum_{l=1}^\infty (l+1)A_l \frac{P_l(\cos\theta)}{R^{l+2}}\bigg]
    = \varepsilon_2\sum_{l=0}^\infty lB_lR^{l-1}P_l(\cos\theta)
\end{align*}
So as we did in Exercise 3 let us multiply the equation by $P_n(\mu)$ where
$\mu =\cos\theta$ and let us integrate with respect to $\mu$ between -1 and 1
\begin{align*}
    &\varepsilon_1\bigg[
    -E_0\int_{-1}^1 \mu P_n(\mu)~d\mu
    - \sum_{l=1}^\infty (l+1)\frac{A_l}{R^{l+2}} \int_{-1}^1P_l(\mu)P_n(\mu)~d\mu
    \bigg]\\
    &\quad= \varepsilon_2\sum_{l=0}^\infty lB_lR^{l-1}\int_{-1}^1 P_l(\mu)P_n(\mu)~d\mu
\end{align*}
Then setting $n = 1$ we get that
\begin{align*}
    \varepsilon_1\bigg[
    -E_0\int_{-1}^1 \mu P_1(\mu)~d\mu
    - \frac{2A_1}{R^{3}} \int_{-1}^1P_1(\mu)P_1(\mu)~d\mu
    \bigg]
    &= \varepsilon_2 B_1 \int_{-1}^1 P_1(\mu)P_1(\mu)~d\mu\\
    \varepsilon_1\bigg[
    -E_0\int_{-1}^1 \mu^2~d\mu - \frac{2A_1}{R^{3}}\frac{2}{3}
    \bigg] &= \varepsilon_2 B_1 \frac{2}{3}\\
    \varepsilon_1\bigg[-E_0\frac{2}{3} - \frac{2A_1}{R^{3}}\frac{2}{3}\bigg]
    &= \varepsilon_2 B_1 \frac{2}{3}\\
    \varepsilon_1\bigg[-E_0 - \frac{2A_1}{R^{3}}\bigg] &= \varepsilon_2 B_1
\end{align*}
If $n = 2$ we get that
\begin{align*}
    \varepsilon_1\bigg[
    -E_0\int_{-1}^1 \mu P_2(\mu)~d\mu
    - \frac{3A_2}{R^{4}} \int_{-1}^1P_2(\mu)P_2(\mu)~d\mu
    \bigg]
    &= 2\varepsilon_2 B_2 R\int_{-1}^1 P_2(\mu)P_2(\mu)~d\mu\\
    %
    \varepsilon_1\bigg[
    -\frac{E_0}{2}\int_{-1}^1 \mu(3\mu^2 - 1)~d\mu
    - \frac{3A_2}{R^{4}}\frac{2}{5}
    \bigg]
    &= 2\varepsilon_2 B_2 R\frac{2}{5}\\
    %
    \varepsilon_1\bigg[
    -\frac{E_0}{2}\bigg[\frac{3\mu^4}{4} - \frac{\mu^2}{2}\bigg]_{-1}^1
    - \frac{3A_2}{R^{4}}\frac{2}{5}\bigg]
    &= 2\varepsilon_2 B_2 R\frac{2}{5}\\
    %
    \varepsilon_1\bigg[-\frac{3A_2}{R^{4}}\bigg]
    &= 2\varepsilon_2 B_2 R
\end{align*}
And if $n = 3$
\begin{align*}
    \varepsilon_1\bigg[
    -E_0\int_{-1}^1 \mu P_3(\mu)~d\mu
    - \frac{4A_3}{R^{5}} \int_{-1}^1P_3(\mu)P_3(\mu)~d\mu
    \bigg]
    &= 3\varepsilon_2 B_3 R^2\int_{-1}^1 P_3(\mu)P_3(\mu)~d\mu\\
    %
    \varepsilon_1\bigg[
    -\frac{E_0}{2}\int_{-1}^1 \mu(5\mu^3 - 3\mu)~d\mu
    - \frac{4A_3}{R^{5}}\frac{2}{7}
    \bigg]
    &= 3\varepsilon_2 B_3 R^2\frac{2}{7}\\
    %
    \varepsilon_1\bigg[
    -\frac{E_0}{2}\bigg[\mu^5 - \mu^3\bigg]_{-1}^1
    - \frac{4A_3}{R^{5}}\frac{2}{7}\bigg]
    &= 3\varepsilon_2 B_3 R^2\frac{2}{7}\\
    %
    \varepsilon_1\bigg[-\frac{4A_3}{R^{5}}\bigg]
    &= 3\varepsilon_2 B_3 R^2
\end{align*}
We can continue this process in the same way for any value of $l$.
\end{proof}

\cleardoublepage
\begin{proof}{\textbf{Exercise 5.}}
From equations (43) and (45) for $A_1$ and $B_1$ we know that
\begin{align*}
    -E_0R + \frac{A_1}{R^2} = B_1 R
    \quad\text{and}\quad
    B_1 = \frac{\varepsilon_1}{\varepsilon_2}\bigg(-E_0 - \frac{2A_1}{R^3}\bigg)
\end{align*}
Then replacing $B_1$ in the first equation we get that
\begin{align*}
    -E_0R + \frac{A_1}{R^2}
    &= \frac{\varepsilon_1}{\varepsilon_2}\bigg(-E_0R - \frac{2A_1}{R^2}\bigg)\\
    -E_0R\bigg(1 - \frac{\varepsilon_1}{\varepsilon_2}\bigg)
    + \frac{A_1}{R^2}\bigg(1 + \frac{2\varepsilon_1}{\varepsilon_2}\bigg)
    &= 0\\
    A_1\bigg(\frac{\varepsilon_2 + 2\varepsilon_1}{\varepsilon_2}\bigg)
    &= E_0R^3\bigg(\frac{\varepsilon_2 - \varepsilon_1}{\varepsilon_2}\bigg)\\
    A_1
    &= E_0R^3\frac{\varepsilon_2 - \varepsilon_1}{\varepsilon_2 + 2\varepsilon_1}
\end{align*}
Now, we replace this value of $A_1$ in the equation we have for $B_1$ as follows
\begin{align*}
    B_1 &= \frac{\varepsilon_1}{\varepsilon_2}
    \bigg(-E_0 - \frac{2E_0R^3}{R^3}
    \frac{\varepsilon_2 - \varepsilon_1}{\varepsilon_2 + 2\varepsilon_1}\bigg)\\
    B_1 &= -E_0\frac{\varepsilon_1}{\varepsilon_2}
    \bigg(1 +
    \frac{2(\varepsilon_2 - \varepsilon_1)}{\varepsilon_2 + 2\varepsilon_1}\bigg)\\
    B_1 &= -E_0
    \bigg(\frac{\varepsilon_1}{\varepsilon_2} + 
    \frac{2\varepsilon_1(\varepsilon_2 - \varepsilon_1)}
    {\varepsilon_2(\varepsilon_2 + 2\varepsilon_1)}\bigg)\\
    B_1 &= -E_0
    \bigg(
    \frac{\varepsilon_1(\varepsilon_2 + 2\varepsilon_1)
    + 2\varepsilon_1(\varepsilon_2 - \varepsilon_1)}
    {\varepsilon_2(\varepsilon_2 + 2\varepsilon_1)}\bigg)\\
    B_1 &= -E_0
    \bigg(\frac{3\varepsilon_1\varepsilon_2}
    {\varepsilon_2(\varepsilon_2 + 2\varepsilon_1)}\bigg)\\
    B_1 &= -E_0
    \bigg(\frac{3\varepsilon_1}{\varepsilon_2 + 2\varepsilon_1}\bigg)
\end{align*}
For $A_2$ and $B_2$ we see that
\begin{align*}
    \frac{A_2}{R^3} = B_2 R^2
    \quad\text{and}\quad
    B_2 = \frac{\varepsilon_1}{\varepsilon_2}\bigg(\frac{-3A_2}{2R^5}\bigg)
\end{align*}
Then replacing $B_2$ we get that
\begin{align*}
    \frac{A_2}{R^3}
    &= R^2\frac{\varepsilon_1}{\varepsilon_2}\bigg(\frac{-3A_2}{2R^5}\bigg)\\
    A_2 &= -\frac{3\varepsilon_1A_2}{2\varepsilon_2}\\
    A_2\bigg( 1 + \frac{3\varepsilon_1}{2\varepsilon_2}\bigg) &= 0\\
    A_2 &= 0
\end{align*}
And hence $B_2 = 0$.
\\
Fianlly, since the equations of $A_i, B_i$ where $i \geq 2$ have the same form,
we get that
\begin{align*}
    A_2 = A_3 = ... = 0\\
    B_2 = B_3 = ... = 0\\
\end{align*}
\end{proof}

\cleardoublepage
\begin{proof}{\textbf{Exercise 6.}}
We know that the total surface charge density on the interface is
\begin{align*}
    \sigma_P = \bigg(1 - \frac{1 - 1/\varepsilon_1}{1 - 1/\varepsilon_2}\bigg)
    \frac{\varepsilon_2 - 1}{4\pi}
    \frac{3\varepsilon_1}{\varepsilon_2 + 2\varepsilon_1}\cos\theta
\end{align*}
We want to consider the special case of an empty spherical cavity
($\varepsilon_2 = 1$), but first we need to re-write the equation as follows
\begin{align*}
    \sigma_P &=\bigg(1 - \frac{\varep_2(\varep_1 - 1)}{\varep_1(\varep_2 - 1)}\bigg)
    \frac{\varepsilon_2 - 1}{4\pi}
    \frac{3\varepsilon_1}{\varepsilon_2 + 2\varepsilon_1}\cos\theta\\
    \sigma_P &=\bigg(
    \varep_2 - 1 - \frac{\varep_2(\varep_1 - 1)}{\varep_1}\bigg)
    \frac{1}{4\pi}
    \frac{3\varepsilon_1}{\varepsilon_2 + 2\varepsilon_1}\cos\theta
\end{align*}
So now, replacing $\varep_2 = 1$ we get that
\begin{align*}
    \sigma_P &= \bigg(
    1 - 1 - \frac{\varep_1 - 1}{\varep_1}\bigg)
    \frac{1}{4\pi}
    \frac{3\varepsilon_1}{1 + 2\varepsilon_1}\cos\theta\\
    &= - \frac{\varep_1 - 1}{\varep_1}\frac{1}{4\pi}
    \frac{3\varepsilon_1}{1 + 2\varepsilon_1}\cos\theta\\
    &= - \frac{1}{4\pi}
    \frac{3(\varep_1 - 1)}{1 + 2\varepsilon_1}\cos\theta
\end{align*}
\end{proof}
\end{document}